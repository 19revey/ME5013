\documentclass[xcolor=pst,serif]{beamer}
\usetheme{utsa}


\addbibresource{/Users/john/Documents/LaTeX/include/all.bib}



\begin{document}

%%%%%% Enter Title Information Here %%%%%%

\title[*nix Commands--grep]{Basic *nix Commands -- grep}
\subtitle{ME 4953/5013 - Introduction to High-Performance Computing}
\author[ME 4953/5013]{}
%\institute{}
%\phone{505-284-4267}
%\email{jtfoste@sandia.gov}
\date{}

%%%%%% Title Frame %%%%%%%
%\section*{Title Page}                             
                                 
\frame{\titlepage}

%%%%%% Begin entering slide content here %%%%

%  find...
%%%%%%%%%%%%%%%%%%%%%%%%%%%%
%

%\section*{}
\frame[t]{
	\frametitle{\tt grep}
	
	\begin{itemize}
		\item \underline{\bf g}lobal \underline{\bf r}egular \underline{\bf e}xpresion \underline{\bf p}rint 
		 \item  {\tt grep} scans its input for a pattern, and can display the selected pattern, the line numbers, or the filenames where pattern occurs.
	\end{itemize}
	
	\begin{block}{Structure of {\tt grep} command, e.g.}
	
	${\tt grep} \underbrace{\tt -i}_{\text{Options}} \quad \overbrace{\text{\tt pat}}^{\text{Patterm}} \quad \underbrace{\text{\tt *.c}}_{\text{filename(s)}}$
	
	\end{block}
	
%	
}

%  Selection criterion...
%%%%%%%%%%%%%%%%%%%%%%%%%%%%
%

%\section*{}
\frame[t]{
	\frametitle{{\tt grep} options}
	
	  %
	\begin{table}[]
		\label{table:second}
		\centering
		\begin{tabular}{l l }
		\rowcolor{UTSAblue} {\color{UTSAorange}Option} & {\color{UTSAorange}Significance} \\
		\rowcolor{UTSAorange!20} {\tt -i}  & Ignores cas for matching \\
		\rowcolor{UTSAorange} {\tt -v}  & Doesn't display line matching expression \\
		\rowcolor{UTSAorange!20} {\tt -n}  & Displays line numbers along with lines\\
		\rowcolor{UTSAorange} {\tt -c}  & Displays count of number of occurrences \\
		\rowcolor{UTSAorange!20} {\tt -l }  & Displays list of filenames only \\
		\rowcolor{UTSAorange} {\tt -e exp}  & Displays {\tt exp} with this option.  Can be \\
		\rowcolor{UTSAorange} & used multiple times. Also used for expression \\
		\rowcolor{UTSAorange} & beginning with a hyphen.\\
		\rowcolor{UTSAorange!20} {\tt -x}  & Matches pattern with entire line \\
		\rowcolor{UTSAorange} {\tt -f file}  & Takes patterns from a file, one \\
		\rowcolor{UTSAorange}   & per line \\
		\end{tabular}
	\end{table}
	
%	
}

%  Action...
%%%%%%%%%%%%%%%%%%%%%%%%%%%%
%

%\section*{}
\frame[t]{
	\frametitle{Comments}
	
	  %
	\begin{itemize}
		\item You can use \emph{regular expressions} in patterns and filenames, e.g.
		\begin{itemize}
			\item {\tt grep ``wo[od][de]house'' *.c}
			\item Will print all matches to either {\tt woodhouse} or {\tt wodehouse} in all the C program files located in the current directory.
		\end{itemize}
		\item Works well with pipe ({\tt |}), e.g.
		\begin{itemize}
			\item {\tt ls /usr/bin | grep gcc}
			\item Prints all executables in {\tt /usr/local} with {\tt gcc} in the command name.
		\end{itemize}
		\item Works well with ({\tt -exec (-ok)}) command with {\tt find},
		\begin{itemize}
			\item Use {\tt find} to find a file matching some regular expression then use {\tt grep} to look inside the file for an additional pattern.
		\end{itemize}
		\item{See also:  the Perl script {\tt ack}}
		\begin{itemize}
			\item A better {\tt grep}.
		\end{itemize}
	\end{itemize}
	
	
%	
}

\end{document}
